\documentclass[a4paper,11pt]{jsarticle}


% 数式
\usepackage{amsmath,amsfonts}
\usepackage{bm}
% 画像
\usepackage[dvipdfmx]{graphicx}


\begin{document}

\title{レポート課題}
\author{古城隆人}
\date{\today}
\maketitle

\newpage
私はsdgsの目標のうち、4番目の「質の高い教育をみんなに」について考える。
教育を受ける権利は世界人権宣言にて、
「すべての人は教育を受ける権利を有する。教育は無償でなければならない。少なくとも初等教育及び基礎教育は無償でなければならない。」\cite{zinken}
と定められている。世界人権宣言とは、1948年に国連総会で採択された国際的な人権の基本的な原則を示した文書である。
このような法律があるにもかかわらず、世界の教育の格差が激しかったり、教育を受けることができない人がいるという問題があるため、重要な問題と考える。
そもそも教育を受けられないということは、その人が持つ可能性を無駄にしているとも言える。世界にある先人が残した情報を学ぶことにより技術が向上したり、
政治がいいものへとなって結果的に世の中がより良いものになっていく。
また、知識がないと詐欺にあったりだまされやすくなるし、低賃金/低所得の状態が悪いと気付かないため経済格差の助長につながる。教育の充実は国家の発展へとつながるのである。\\
しかし、教育によって経済格差が縮まることはない。現に日本では義務教育という制度があり、政府の公認教科書を使ってすべての人が質の高い教育を受けられていると考えるが、
日本の経済格差は"広がり切った"と評価されている。経済格差の一つの指標であるジニ係数は近年では横ばいの状態が続いているが、資産では広がり続けている。
資本主義の体現によって日本の経済格差が広がっていると考えられる。このことは資本主義の課題の一つであるが、話がそれるため割愛する。\\
\\
この目標が達成されるためには学校以外のインフラの改善が必要と考える。JICAやユニセフなどは途上国に対して教育支援を行っているが、それは教育をするための環境の整備だけである。
学校にいけない理由として、家庭的・経済的な理由がある。家庭的な理由としては小さい子供でも働き手として働かないといけなかったり、赤ちゃんのお世話をするために学校にいけない理由がある。
経済的な理由としては、紛争地域ではそもそも環境が整えられてなかったり少年兵として子供が使われるため学校にいけないのである。\\
この目標に近づくために行えることは、まずは教育を受けられない理由を取り除くことである。
そのためには、先進国による支援を行い、新しい制度として初等教育手当を支給させるといいと考える。
子供が学業によって本来生じたはずの時間に応じた手当を"子供"に給付し、そのお金を子供に使わせるようにさせる。これをすることにより、経済についての知識が子供につくだけでなく、いろいろなところにお金を使用するため経済成長を見込むことが出来る。
経済が成長することにより政府の税収が増え、まっとうな使い方をされれば公共の福祉の充実につながるともいえる。しかし、本来の想定している使い方をされないと改善が期待できい。
そのために監査機関の設置も同時に行うべきだと考える。\\
これらの効果を期待するためには先進国の多大な出費が伴うため、先進国の各企業が地域に工場などの"インフラ"を作るとともに、その地域の教育に対して出資を行う。
これにより企業は利益の拡大をさせることが出来るし、経済が大きくなることにより新たな市場を作ることが出来る。それによって企業は利益を伸ばすことが出来るし、質の高い教育を展開することが容易になる。また、政府の税収も増えるため、公共の福祉の充実にもつながる。\\
この方法では先進国による搾取になりえる可能性があるため、その地域の人々による監査機関の設置も行うべきだと考える。その監査機関の権力を大きくするためには発展国の政府による監査機関の設置を行う必要があると考える。\\
以上のような方法を取ることにより、質の高い教育を受けられる環境を整えることが出来ると考える。しかし、紛争地域の経済的な事情には依然として問題が残っている。
そこで、

\begin{thebibliography}{99}
  \bibitem{zinken} 外務省 世界人権宣言 和訳,https://www.mofa.go.jp/mofaj/gaiko/udhr/1b\_002.html,参照日(2024年7月18日)
  % 他の文献もここに追加する
\end{thebibliography}  

\end{document}

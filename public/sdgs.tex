\documentclass[a4paper,11pt]{jsarticle}


% 数式
\usepackage{amsmath,amsfonts}
\usepackage{bm}
% 画像
\usepackage[dvipdfmx]{graphicx}


\begin{document}

\title{レポート課題}
\author{古城隆人}
\date{\today}
\maketitle

\newpage
私はsdgsの目標のうち、4番目の「質の高い教育をみんなに」について考える。
教育を受ける権利は世界人権宣言にて、
「すべての人は教育を受ける権利を有する。教育は無償でなければならない。少なくとも初等教育及び基礎教育は無償でなければならない。」\cite{zinken}
と定められている。世界人権宣言とは、1948年に国連総会で採択された国際的な人権の基本的な原則を示した文書である。
このような法律があるにもかかわらず、世界の教育の格差が激しかったり、教育を受けることができない人がいるという問題があるため、重要な問題と考える。
そもそも教育を受けられないということは、その人が持つ可能性を無駄にしているとも言える。世界にある先人が残した情報を学ぶことにより技術が向上したり、
政治がいいものへとなって結果的に世の中がより良いものになっていく。
また、知識がないと詐欺にあったりだまされやすくなるし、低賃金/低所得の状態が悪いと気付かないため経済格差の助長につながる。教育の充実は国家の発展へとつながるのである。\\
しかし、教育によって経済格差が縮まることはない。現に日本では義務教育という制度があり、政府の公認教科書を使ってすべての人が質の高い教育を受けられていると考えるが、
日本の経済格差は"広がり切った"と評価されている。経済格差の一つの指標であるジニ係数は近年では横ばいの状態が続いているが、資産では広がり続けている。
資本主義の体現によって日本の経済格差が広がっていると考えられる。このことは資本主義の課題の一つであるが、話がそれるため割愛する。\\
\\
この目標が達成されるためにはすでにノウハウがある国からの支援が必須と考える。教育の質が低い国というのは教育の方法やノウハウがないと考え、それらを自分たちで研究するよりも
違う国から購入という形で取り入れることにより、質の高い教育を早く導入できると考える。

\begin{thebibliography}{99}
  \bibitem{zinken} 外務省 世界人権宣言 和訳,https://www.mofa.go.jp/mofaj/gaiko/udhr/1b\_002.html,参照日(2024年7月18日)
  % 他の文献もここに追加する
\end{thebibliography}  

\end{document}
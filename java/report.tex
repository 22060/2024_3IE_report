\documentclass[a4paper,11pt]{ltjsarticle}
% 数式
\usepackage{amsmath,amsfonts}
\usepackage{bm}
% 画像
\usepackage{graphicx}
\usepackage{circuitikz}
\usepackage{amsmath,amssymb}
\usepackage{siunitx}
\usepackage{float}
\usepackage{tikz}
\usepackage{askmaps}
\usepackage{multirow}
\usepackage{bigstrut}
\usepackage{slashbox}
\usepackage{rotating}
\usepackage{listings}
% 数式
\usepackage{physics}
\usepackage{mathtools}
% 画像
\usepackage{subcaption}
% 表
\usepackage{makecell}
% その他
\usepackage{url}
\usepackage{ascmac}
\usepackage{cases}
\usepackage{here}
\usepackage{upgreek}
% 日本語対応
\usepackage{luatexja}
\usepackage{luatexja-fontspec}

\AtBeginDocument{\RenewCommandCopy\qty\SI}


\definecolor{commentgreen}{RGB}{0,200,0}
\definecolor{eminence}{RGB}{120,80,250}
\definecolor{weborange}{RGB}{255,165,0}
\definecolor{frenchplum}{RGB}{10,150,200}
\definecolor{commentgreen}{RGB}{0,200,0}
\definecolor{eminence}{RGB}{120,80,250}
\definecolor{weborange}{RGB}{255,165,0}
\definecolor{frenchplum}{RGB}{10,150,200}

\lstset{
        language = {C},
        basicstyle = \ttfamily\small,
        keywordstyle=\color{eminence}\ttfamily\bfseries,
        commentstyle=\color{commentgreen}\textit,
    identifierstyle=\color{black}\ttfamily,
        xleftmargin=.35in,
        frame=lines,
    showstringspaces=false,
        numbers=left,
        stepnumber = 1,
        breaklines=true,
        numberstyle = \ttfamily\normalsize,
    tabsize=4,  
        emph={int, int8_t, int16_t, int32_t, int64_t, uint8_t, uint16_t, uint32_t, uint64_t, char, double, float, unsigned, void, bool},
        emphstyle={\color{blue}}, 
        morekeywords={>, <, ., ;, +, -, *, /, !, =, ~},
        breakindent = 10pt, 
        framexleftmargin=10mm, 
        columns=fixed,
        basewidth=0.5em,
        }

% 特定のスタイル設定
\lstdefinestyle{customtxt}{
  basicstyle=\ttfamily\footnotesize,
  backgroundcolor=\color{lightgray},
  frame=single,
  breaklines=true,
  columns=fullflexible,
  showspaces=false,
  showstringspaces=false,
  showtabs=false,
  tabsize=4,
}

\newcommand{\fig}[4]{
    \begin{figure}[htbp]
      \centering
      \includegraphics{./image/#1}
      \caption{#2}
      \label{fig:#3}
    \end{figure}
  }
\begin{document}

\title{キュークラス}
\author{古城隆人}
\date{\today}
\maketitle
\newpage


\begin{lstlisting}[caption={main class},label={lst:label}]
    public class newClass {

	public static void main(String[] args) {
		// Queue.defaultSize = 10;
		Queue queue = new Queue();
		System.out.println(queue.QueueVolume() + "個目のスタックです");
		Queue tower10 = new Queue(10);
		System.out.println(queue.QueueVolume() + "個目のスタックです");
		tower10 = new Queue();
		// System.out.println(queue.queueCount + "個目のスタックです");
		queue.enqueue(10);queue.printQueue();
		queue.enqueue(20);queue.printQueue();
		// queue.volemu = 0;
		queue.enqueue(30);queue.printQueue();
		queue.enqueue(40);queue.printQueue();
		queue.enqueue(50);queue.printQueue();
		queue.enqueue(60);queue.printQueue();
		// queue.data[0] = 100;
		// queue.printQueue(0);
		System.out.println(queue.dequeue());queue.printQueue(3,4);queue.printQueue();
		System.out.println(queue.dequeue());queue.printQueue();
		System.out.println(queue.dequeue());queue.printQueue();
		System.out.println(queue.dequeue());queue.printQueue();
		System.out.println(queue.dequeue());queue.printQueue();
		System.out.println(queue.dequeue());queue.printQueue();
	}

}
\end{lstlisting}

\begin{lstlisting}[caption={Queue class},label={lst:label1}]
    
public class Queue {
	private int volume;
	private int data[];
	private final static int defaultSize = 5;
	private static int queueCount = 0;

	Queue() {
		this(defaultSize);
	}

	Queue(int n) {
		data = new int[n];
		System.out.println(data.length + "個分のキュー生成");
		queueCount++;
		// System.out.println(queueCount + "個目のスタックです");
	}

	int enqueue(int number) {
		int value;
		// 残容量確認
		if (data.length > volume) {
			// 入力値確認
			if (number > 0) {
				data[volume] = number;
				volume++;
				value = 1;
			} else {
				value = 0;
				System.out.println("wrong input");
			}
		} else {
			System.out.println("queue overflow");
			value = 0;
		}

		return value;
	}

	// データ取得関数
	int dequeue() {
		int value;
		// 格納個数確認
		if (volume > 0) {
			value = data[0];
			volume--;

			// 空き領域を埋めるためのシフト
			for (int i = 0; i < data.length - 1; i++) {
				data[i] = data[i + 1];
			}
			data[volume] = 0;
		} else {
			value = -1;
		}
		return value;
	}

	// 状態表示関数
	void printQueue() {
		this.printQueue(0, data.length);
		// System.out.print("|");
		// for (int i = 0; i < data.length; i++) {
		// 	System.out.printf("%d", data[i]);
		// 	System.out.printf("|");
		// }
		// System.out.println();
	}

	private void printQueue(int num){
		System.out.print(data[num]);
	}

	// 状態表示関数
	void printQueue(int start, int end) {
		System.out.print("|");
		for (int i = start; i < end; i++) {
			printQueue(i);
			System.out.printf("|");
		}
		System.out.println();
	}

	int QueueVolume() {
		return queueCount;
	}
}

\end{lstlisting}

\begin{lstlisting}[caption={実行結果},label={lst:label2}]
5個分のキュー生成
1個目のスタックです
10個分のキュー生成 
2個目のスタックです
5個分のキュー生成  
|10|0|0|0|0|       
|10|20|0|0|0|      
|10|20|30|0|0|     
|10|20|30|40|0|    
|10|20|30|40|50|   
queue overflow     
|10|20|30|40|50|   
10
|50|
|20|30|40|50|0|    
20
|30|40|50|0|0|     
30
|40|50|0|0|0|      
40
|50|0|0|0|0|       
50
|0|0|0|0|0|        
-1
|0|0|0|0|0| 
\end{lstlisting}


\end{document}
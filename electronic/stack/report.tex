\documentclass[a4paper,11pt]{jsarticle}


% 数式
\usepackage{amsmath,amsfonts}
\usepackage{bm}
% 画像
\usepackage[dvipdfmx]{graphicx}

\include{setting.txt}


\begin{document}

\title{スタックの実装と応用}
\author{古城隆人}
\date{\today}
\maketitle

\begin{table}[b]
  \centering
  \begin{tabular}{:c:r:}
    \hdashline
    報告者 & 22060 古城 隆人 \\
    \hdashline
    提出日 & \today \\
    \hdashline
  \end{tabular}
\end{table}

\tableofcontents

\newpage


\section{目的}
ススタックを用いたデータ構造を理解し、スタックを用いたプログラムを作成することで、スタックの基本的な操作を理解する。また、スタックを用いたハノイの塔のプログラムを作成し、理解を深める。
\section{原理}
\subsection{スタック}
スタックとはデータ構造の一種であり、データの追加や削除が限定的に行うことが出来る構造である。
限定的にすることによりデータの数が増えてもデータの挿入や削除にかかる時間が一定になるという利点がある。
スタックはデータが出入りする順番により、LIFO(Last In First Out)、FILO(First in Last out)と呼ばれる。
スタックには以下の操作がある。
\begin{itemize}
  \item push: スタックにデータを追加する
  \item pop: スタックからデータを取り出す
\end{itemize}
pushを行うと、スタックの一番上にデータが追加される。popを行うと、スタックの一番上のデータが取り出される。
\subsection{ハノイの塔}
ハノイの塔は、3本の杭とその上に積まれた円盤からなるパズルである。移動のルールは以下の通りである。
\begin{itemize}
  \item 1回の移動で1枚の円盤しか動かせない
  \item 小さい円盤の上に大きい円盤を乗せることはできない
  \item すべての円盤を移動させるとき、最初の杭から最後の杭に移動させる
\end{itemize}
\section{実験環境}
\begin{table}[ht]
  \centering
  \begin{tabular}{|c|c|}
    \hline
    \textbf{項目} & \textbf{値}              \\
    \hline
    OS          & windows10上のwsl2(Ubuntu) \\
    \hline
    CPU         & Intel Core i7           \\
    \hline
    メモリ         & 8GB                     \\
    \hline
    コンパイラ       & gcc 11.4.0              \\
    \hline
  \end{tabular}
  \caption{実験環境}
  \label{tab:environment}
\end{table}

\section{プログラムの設計と説明}
\subsection{スタックの実装}
スタックの実装には、配列を用いる。スタックの構造体には、スタックのサイズ、スタックの配列を持つ。
\subsubsection{構造体}
\begin{lstlisting}[caption={stack.h}, label={lst:stackh}]
#define HEIGHT 5
struct Stack
{
  int data[HEIGHT];
  int volume;
};
\end{lstlisting}
\subsubsection{作成する関数}
\begin{itemize}
  \item init: スタックの構造体を初期化する、関数の仕様は 表\ref{tab:init_func}に示す。
        \begin{table}[ht]
          \centering
          \begin{tabular}{|c|c|}
            \hline
            機能  & スタック構造体の中にある配列を全部0で初期化する。            \\
            \hline
            引数  & struct Stack *stack : 初期化するスタックのポインタ \\
            \hline
            戻り値 & なし                                   \\
            \hline
          \end{tabular}
          \caption{init関数}
          \label{tab:init_func}
        \end{table}
  \item pop: スタックからデータを取り出す、関数の仕様は 表\ref{tab:pop_func}に示す。
  \begin{table}[ht]
    \centering
    \begin{tabular}{|c|c|}
      \hline
      機能  & スタックのデータを取り出して、そのデータを削除する。            \\
      \hline
      引数  & struct Stack *stack : popしたいstackのポインタ \\
      \hline
      戻り値 & 削除したデータの値(int)を返すが、エラーが出たときは-1(int)を返す                               \\
      \hline
    \end{tabular}
    \caption{pop関数}
    \label{tab:pop_func}
  \end{table}
  \item push: スタックに値を代入する、関数の仕様は 表\ref{tab:push_func}に示す。
  \begin{table}[ht]
    \centering
    \begin{tabular}{|c|c|}
      \hline
      機能  & スタックにデータを追加させる。            \\
      \hline
      引数  & struct Stack *stack : pushしたいstackのポインタ \\
      \hline
      戻り値 & 処理が正常に終了したら1、エラーが出たら-1                               \\
      \hline
    \end{tabular}
    \caption{push関数}
    \label{tab:push_func}
  \end{table}
  \newpage
  \item printStack: スタックの中身を表示する、関数の仕様は 表\ref{tab:printstack_func}に示す。
  \begin{table}[ht]
    \centering
    \begin{tabular}{|c|c|}
      \hline
      機能  & スタックの中身を表示する。            \\
      \hline
      引数  & struct Stack *stack : 表示したいstackのポインタ \\
      \hline
      戻り値 & なし                               \\
      \hline
    \end{tabular}
    \caption{printStack関数}
    \label{tab:printstack_func}
  \end{table}
  \item pushTest: push関数のテストを行う、関数の仕様は 表\ref{tab:pushtest_func}に示す。
  \begin{table}[ht]
    \centering
    \begin{tabular}{|c|c|}
      \hline
      機能  & push関数のテストを行う。成功したらpush関数の戻り値が0になるため、\\ &if文を使い結果を出力する           \\
      \hline
      引数  & struct Stack *stack : pushしたいstackのポインタ \\
       & num: pushする値 \\
      \hline
      戻り値 & なし                               \\
      \hline
    \end{tabular}
    \caption{pushTest関数}
    \label{tab:pushtest_func}
  \end{table}
  \item popTest: pop関数のテストを行う、関数の仕様は 表\ref{tab:poptest_func}に示す。
  \begin{table}[ht]
    \centering
    \begin{tabular}{|c|c|}
      \hline
      機能  & pop関数のテストを行う。popに失敗したらpop関数の戻り値が-1になるため、\\ & if文を使い結果を出力する            \\
      \hline
      引数  & struct Stack *stack : popしたいstackのポインタ \\
      \hline
      戻り値 & なし                               \\
      \hline
    \end{tabular}
    \caption{popTest関数}
    \label{tab:poptest_func}
  \end{table}
\end{itemize}

\subsection{ハノイの塔}
\section{プログラムの説明}
\section{実行結果}
\subsection{スタックの実装}
\subsection{ハノイの塔}
\section{考察}
\section{付録:今回使用したプログラム}
\lstinputlisting[caption={main.c}, label={lst:main}]{stack.c}


\end{document}
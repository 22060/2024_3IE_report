\documentclass[dvipdfmx]{jsarticle}
\usepackage[final]{graphicx}

% 数式
\usepackage{amsmath,amsfonts}
\usepackage{bm}



\usepackage{listings, jlisting}


\usepackage{xcolor}

%ソースコードの色
\definecolor{commentgreen}{RGB}{0,200,0}
\definecolor{eminence}{RGB}{120,80,250}
\definecolor{weborange}{RGB}{255,165,0}
\definecolor{frenchplum}{RGB}{10,150,200}

\lstset{
        language = {C},
        basicstyle = \ttfamily\small,
        keywordstyle=\color{eminence}\ttfamily\bfseries,
        commentstyle=\color{commentgreen}\textit,
    identifierstyle=\color{black}\ttfamily,
        xleftmargin=.35in,
        frame=lines,
    showstringspaces=false,
        numbers=left,
        stepnumber = 1,
        breaklines=true,
        numberstyle = \ttfamily\normalsize,
    tabsize=4,  
        emph={int, int8_t, int16_t, int32_t, int64_t, uint8_t, uint16_t, uint32_t, uint64_t, char, double, float, unsigned, void, bool},
        emphstyle={\color{blue}}, 
        morekeywords={>, <, ., ;, +, -, *, /, !, =, ~},
        breakindent = 10pt, 
        framexleftmargin=10mm, 
        columns=fixed,
        basewidth=0.5em,
        }

        

\begin{document}

\title{ファイル操作}
\author{古城隆人}
\date{\today}
\maketitle


\newpage
\begin{center}
  \begin{center}
    \Huge ファイル操作とコマンド引数
  \end{center}
\end{center}


\section{課題の内容}
\subsection{課題1}
キーボードから入力された文字列を書き込むプログラムを作成する。要件は以下のとおりである。
\begin{itemize}
  \item 書き込むファイル名はアプリの起動時に指定する。
  \item オプションとして-iがある場合があり、これが指定された場合には、書き込むファイルの有無を調べる、もしすでにファイルがあったら上書きしてよいか問い合わせる、その結果、上書きしてはいけない場合には処理を終了する。
  \item キーボードからエンターキーを押されるまでの文字をファイルに書き込む。
  \item 上記ののちに終了するか問い合わせ、終了しないならもう一度上記の処理を行う。
\end{itemize}
\subsection{課題2}
シュミレーション結果をファイルに書き込むプログラムを作成する。要件は以下のとおりである。
\begin{itemize}
  \item 3桁の数字をランダムに発生させる。
  \item もし3桁がすべて同じ値なら「あたり」、そうでなければ「はずれ」という文字列を発生させる。
  \item 発生させた数字や文字列をファイルに保存する。
  \item 最後に期待値と実際の確率をファイルに書き込む。
  \item 保存ファイル名はmain関数への引数にする。
  \item シミュレーションする回数もmain関数への引数にする。
\end{itemize}
\section{プログラムリスト}
\lstinputlisting[caption={main.c}, label={lst:main}]{main.c}

\section{プログラムの説明}
defineを使用して2個のプログラムを一個のプログラムにまとめた。
\subsection{プログラム1個目}

\subsubsection{main関数}
main関数では、表\ref{table:variables}の変数が宣言されています。\\
また、表\ref{table:arguments}の変数をmain関数の引数として受け取ります。\\
以下プログラムの各行ごとの説明を行う。
\begin{itemize}
  \item 14行目のif文でモードの判別を行っている。
  \item 15行目で指定されたファイルが存在するかを確認し、ない場合はNULLとなるため次のif文でtrueとなる。
  \item 20行目のelse文に入るときは引数で指定したファイルが存在しているときである。
  \item 24行目から27行目ではgoto文を用いて標準入力のバッファーが空になるまで繰り返す。
  \item 29行目のwhile文の条件式にてYかNの入力かを確認している。
  \item 30行目でもし、上書きしてはいけない場合はfileを閉じてプログラムをステータス-1で終了する。
  \item 36行目で-iオプションが指定されてないとみなし、引数が1個であることと-iではないことを確認している。
  \item 37行目で指定されたファイルを読み込むためのポインタを取得している。
  \item 38行目から41行目でファイルが存在しない場合はエラーを出力してプログラムを終了する。
  \item 42行目から46行目では引数の入力errorなので、エラーを出力してプログラムを終了する。
  \item 47行目から56行目ではwhile文を使用してファイルに文字の入力をしている。
\end{itemize}



\begin{table}[ht]
  \centering
  \begin{tabular}{|c|c|c|}
    \hline
    型     & 変数名 & 説明                         \\
    \hline
    FILE* & wfp & 書き込み用のファイルのポインタ変数          \\
    char[100] & input & 入力された変数を保持する変数             \\
    int   & ch  & 標準入力で入力された一文字を保持するための変数 \\
    \hline
  \end{tabular}
  \caption{main関数で宣言されている変数のリスト}
  \label{table:variables}
\end{table}

\begin{table}[ht]
  \centering
  \begin{tabular}{|c|c|c|}
    \hline
    型       & 変数名  & 説明                              \\
    \hline
    int     & argc & メイン関数の引数の数                      \\
    char*[] & argv & メイン関数の引数のchar[]型の配列の先頭アドレスのポインタ \\
    \hline
  \end{tabular}
  \caption{main関数で宣言されている変数のリスト}
  \label{table:arguments}
\end{table}

\subsection{プログラム2個目}
\subsubsection{main関数}
main関数では、表\ref{table:variables2}の変数が宣言されています。\\
また、表\ref{table:arguments2}の変数をmain関数の引数として受け取ります。\\
以下プログラムの各行ごとの説明を行う。

\begin{itemize}
  \item 70行目でsrand関数を呼び出し、今の時間のtimestampをseedとして登録している。
  \item 75行目から78行目ではmain関数の引数が2個であることを確認している。引数が2個でない場合はエラーを出力してプログラムを終了する。
  \item 79行目から83行目では引数で指定されたファイルを読み込むためのポインタを取得している。
  \item 84行目ではcount変数に引数で指定された回数を代入している。atoi関数は文字列を数値に変換する関数である。
  \item 85行目から95行目ではfor文を使用してcount回数だけ乱数を生成している。乱数が3桁とも同じ数字である場合はj変数をインクリメントする。また、fprintf関数を使用してファイルに乱数と結果を書き込んでいる。
  \item 96行目では試行回数、あたり回数、期待値、確率をファイルに書き込んでいる。
  \item 99行目ではファイルを閉じてプログラムを終了する。
\end{itemize}


\begin{table}[ht]
  \centering
  \begin{tabular}{|c|c|c|}
    \hline
    型     & 変数名 & 説明                         \\
    \hline
    FILE* & rfp & 書き込み用のファイルのポインタ変数          \\
    int   & ch  & ファイルから読み込んだ文字を保持する変数 \\
    int   & count &  乱数を生成する試行回数\\
    int   & j & 乱数が3桁ともそろった回数を保存する変数\\
    \hline
  \end{tabular}
  \caption{main関数で宣言されている変数のリスト}
  \label{table:variables2}
\end{table}

\begin{table}[ht]
  \centering
  \begin{tabular}{|c|c|c|}
    \hline
    型       & 変数名  & 説明                              \\
    \hline
    int     & argc & メイン関数の引数の数                      \\
    char*[] & argv & メイン関数の引数のchar[]型の配列の先頭アドレスのポインタ \\
    \hline
  \end{tabular}
  \caption{main関数で宣言されている変数のリスト}
  \label{table:arguments2}
\end{table}


\section{結果の説明}
\subsection{プログラム1個目}
ファイルが存在しない場合の実行結果を図\ref{fig:result_pr1_ndup}に示す。\\
ファイルが存在する場合の実行結果を図\ref{fig:result_pr1_dup}に示す。\\
また、ファイルの出力結果を表\ref{table:file1}に示す。
abc1.txtではファイルが存在しないとき、abc2.txtではファイルが存在するときの実行結果である。どちらの場合も同じ引数で実行しているため、場合分けして実行した後にファイル名を変更している。
\begin{table}[ht]
  \centering
  \begin{tabular}{|c|c|}
    \hline
    ファイル名 & 出力結果 \\
    \hline
    \hline
    abc1.txt & \begin{tabular}{c}
    \input{./txt/abc1.txt }  
    \end{tabular}\\
    \hline
    abc2.txt & \begin{tabular}{c}
      \input{./txt/abc2.txt }  
      \end{tabular} \\
    \hline
  \end{tabular}
  \caption{ファイルの出力結果\\(abcのときがファイルが存在しないときで、abc2のときがファイルが存在するとき)}
  \label{table:file1}
\end{table}

\begin{figure}[ht]
  \centering
  \includegraphics[width=0.5\textwidth]{./img/result_pr1_ndup.png}
  \caption{ファイルが存在しないときの実行結果}
  \label{fig:result_pr1_ndup}
\end{figure}

\begin{figure}[ht]
  \centering
  \includegraphics[width=0.5\textwidth]{./img/result_pr1_dup.png}
  \caption{ファイルが存在するときの実行結果}
  \label{fig:result_pr1_dup}
\end{figure}

\newpage
\subsection{プログラム2個目}
ファイルの出力結果を図\ref{fig:result_pr2}に示す。\\

\begin{figure}[ht]
  \centering
  \includegraphics[width=0.5\textwidth]{./img/result_pr2.png}
  \caption{ファイルの出力結果}
  \label{fig:result_pr2}
\end{figure}





\section{考察}
\subsection{プログラム1個目}
ファイルが存在しない場合と存在する場合での実行結果を比較すると、上書きした時に上の行だけでなく、下の行の文字も上書きされていた。
ここから考えられることとしては、
ファイルを読み込んだらその値がメモリに格納されfprintfでそのメモリに干渉して上書きしていると読み取れる。
なのでfprintfやfputcを行っただけではファイルにメモリの値が書き込まれるわけではないと考えられる。
\subsection{プログラム2個目}
乱数を生成してファイルに書き込むプログラムを作成出来たと思う。\\
生成した乱数が3桁とも同じ確率が毎回ぶれていたけど、回数を増やせば確率が1\%の期待値に近づくと思った。

\end{document}